\documentclass[UTF8]{article}
\usepackage[UTF8]{ctex}

\usepackage{xunicode, xltxtra, listings, geometry, indentfirst, xeCJK, amsmath, clrscode, enumerate, indentfirst, wrapfig, color, caption, amssymb, multicol, ulem, esvect, bm, amsthm, extarrows}

\setlength{\parindent}{2em}
%\setmonofont{Menlo}

\newcounter{RomanNumber}
\newcommand{\mrm}[1]{(\setcounter{RomanNumber}{#1}\Roman{RomanNumber})}
\newfontfamily\menlo{Menlo}

\def\e {\mathrm{e}}
\newcommand{\vc}[1]{\overrightarrow{#1}}
\newcommand{\sq}[1]{\sqrt{#1}}
\newcommand{\fr}[2]{\dfrac{#1}{#2}}
\def\st {\mathrm{s.t.}}
\def\eps {\varepsilon}
\def\ds {\displaystyle}
\def\CC {\mathbb{C}}
\def\FF {\mathbb{F}}
\def\GG {\mathbb{G}}
\def\QQ {\mathbb{Q}}
\def\RR {\mathbb{R}}
\def\NN {\mathbb{N}}
\def\ZZ {\mathbb{Z}}
\def\ii {\mathrm{i}}
\def\rtrans {\overset{r}{\rightarrow}}
\def\ctrans {\underset{c}{\rightarrow}}
%\def\r {\mathrm{r}}
\def\cd {\cdot}
\def\LRA{\Leftrightarrow}
\newcommand{\eq}[1]{\xlongequal{\mathrm{#1}}}
\def\smtn{\ds\sum_{i=1}^n}
\newcommand{\sm}[3]{\ds\sum_{#1=#2}^{#3}}
\newcommand{\mttt}[4]{
\left[ \begin{matrix}
		#1 & #2 \\
		#3 & #4
	\end{matrix} \right]
}

\lstset{
language = C++, numbers=left,basicstyle=\linespread{0.8}\menlo,
         numberstyle=\linespread{0.8}\menlo,breaklines=true,
         frame=box,basicstyle=\ttfamily
}

\theoremstyle{definition}
\newtheorem{quu}{题目}
\newtheorem{jl}{结论}
\newcommand{\qu}[1]{\quu{#1}\vspace{5cm}}

\title{不围棋AI设计展示}
\author{董海辰 518030910417}

\begin{document}

\maketitle

\tableofcontents
\newpage

\section{基本信息}

\paragraph{AI代号} White Album 2(下简称WA2)

\paragraph{代码长度} 9.38KB

\paragraph{关键算法} Monte Carlo树搜索及其优化

\section{设计思路}

在传统的Monte Carlo树搜索(MCTS)中,将棋盘状态以落子转移保存为树结构,对于一个节点进行一定次数的随机模拟,来判断一个节点及其父节点的价值(获胜概率).

对于一个局面随机模拟的庞大计算量限制了模拟次数,导致在每次估价过程中没有足够的数据来判断价值,成为MCTS算法的主要瓶颈,因而WA2的主要设计思路即是增加每个节点的样本数量.

\section{具体实现}

\subsection{完整代码}

\begin{lstlisting}
//
//  NoGo.cpp
//  AI_NoGo
//
//  Created by Haichen Dong on 2018/10/23.
//  Copyright © 2018 Haichen Dong. All rights reserved.
//
//  Version ONLINE_JUDGE with TL=1.5s
//

#pragma GCC optimize ("O3")

#include "submit.h"
#include <bits/stdc++.h>
using namespace std;

extern int ai_side;
std::string ai_name = "董海辰";
int PRINTFLAG;
const int TIMELIMIT = CLOCKS_PER_SEC / 20 * 25;
const double K = 1000.0;

struct Status {
    unsigned long long a[3];
    int color, exd, mvx, mvy, lson, rson, fa;
    int n, n1, w, w1;
    map<short,int> son;
    inline Status() {
        son.clear();
        exd = 0;
    }
    inline void getBoard (int board[9][9]) {
        unsigned long long _a[3];
        for (int i = 0; i < 3; i++) _a[i]=a[i];
        for (int k = 0; k < 3; k++) {
            for (int i = 0; i < 3; i++) {
                for (int j = 0; j < 9; j++) {
                    board[i + k*3][j] = _a[k] & 3;
                    _a[k] >>= 2;
                }
            }
        }
    }
    inline void init (int _color,int board[9][9]) {
        color = _color;
        a[0] = a[1] = a[2] = 0;
        for (int k = 2; k >= 0; k--) {
            for (int i = 2; i >= 0; i--) {
                for (int j = 8; j >=0 ; j--) {
                    a[k] <<= 2;
                    a[k] += board[i + k*3][j];
                }
            }
        }
    }
} sTree[(int)1e6];
const int dx[]= {0, 0, 1, -1}, dy[]= {1, -1, 0, 0};
queue< pair<int,int> > q, qq, qqq;
int bk[9][9], ok[9][9], a[9][9], qc[9][9], tmp[9][9], q1, q2;
inline void bfs (int sx, int sy, int type) {
    int qiCnt = 0, qix = 0, qiy = 0;
    bk[sx][sy] = 1;
    while (!q.empty())
        q.pop();
    while (!qq.empty())
        qq.pop();
    while (!qqq.empty())
        qqq.pop();
    q.push(make_pair(sx, sy));
    while (!q.empty()) {
        int cux = q.front().first, cuy = q.front().second;
        qq.push(q.front());
        q.pop();
        for (int k = 0; k < 4; k++) {
            int nex = cux + dx[k], ney = cuy + dy[k];
            if (nex < 0 || nex > 8 || ney < 0 || ney > 8)
                continue;
            if (!a[nex][ney]) {
                if (!tmp[nex][ney]) {
                    tmp[nex][ney] = 1;
                    qiCnt++, qix = nex, qiy = ney;
                    qqq.push(make_pair(nex, ney));
                }
            } else if (a[nex][ney] == a[sx][sy]) {
                if (!bk[nex][ney]) {
                    bk[nex][ney] = 1;
                    q.push(make_pair(nex, ney));
                }
            }
        }
    }
    if (type) {
        q2 += qiCnt;
        if (qiCnt == 1)
            ok[qix][qiy] = 0;
    } else {
        q1 += qiCnt;
        while (!qq.empty()) {
            qc[qq.front().first][qq.front().second] = qiCnt;
            qq.pop();
        }
    }
    while (!qqq.empty()) {
        tmp[qqq.front().first][qqq.front().second] = 0;
        qqq.pop();
    }
}
vector< pair<int,int> > possibleVec, tmpvec;
inline pair<int,int> findPossiblePos (int color, vector< pair<int,int> > &vcr = possibleVec) {
    for (int i = 0; i < 9; i++) {
        for (int j = 0; j < 9; j++) {
            bk[i][j] = 0;
            qc[i][j] = 0;
            if (!a[i][j])
                ok[i][j] = 1;
            else
                ok[i][j] = 0;
        }
    }
    for (int i = 0; i < 9; i++) {
        for (int j = 0; j < 9; j++) {
            if (a[i][j] && !bk[i][j]) {
                bfs(i, j, a[i][j] == 3-color);
            }
        }
    }
    for (int i = 0; i < 9; i++) {
        for (int j = 0; j<9; j++) {
            if (!a[i][j]) {
                int qcu = 0;
                for (int k = 0; k < 4; k ++) {
                    int nex = i + dx[k], ney = j + dy[k];
                    if (nex < 0 || nex > 8 || ney < 0 || ney > 8)
                        continue;
                    if (a[nex][ney] == color) {
                        qcu = qcu + qc[nex][ney] - 1;
                    } else if (!a[nex][ney]) {
                        qcu = 100;
                    }
                }
                if (!qcu)
                    ok[i][j] = 0;
            }
        }
    }
    vcr.clear();
    for (int i = 0; i < 9; i++) {
        for (int j = 0; j < 9; j++) {
            if (ok[i][j])
                vcr.push_back(make_pair(i, j));
        }
    }
    if (!vcr.size())
        return make_pair(-1, -1);
    swap(vcr[rand() % vcr.size()], vcr[0]);
    return vcr[0];
}

inline int simulate (Status s) {
    int curColor = 3 - s.color;
    s.getBoard(a);
    while (1) {
        pair<int,int> pos = findPossiblePos(curColor);
        if (pos.first == -1) {
            return 3 - curColor;
        }
        a[pos.first][pos.second] = curColor;
        curColor = 3 - curColor;
    }
}

double _beta[200005];
inline double cal (int k, int flag) {
    double beta = _beta[sTree[k].n];
    return (1.0 - beta) * sTree[k].w / sTree[k].n + beta * sTree[k].w1 / sTree[k].n1;
}
inline int getBestSon (int k, int flag) {
    double ma = 0;
    int mapo = sTree[k].lson;
    for (int i = sTree[k].lson, rs = sTree[k].rson; i <= rs; i++) {
        double cu = cal(i, flag);
        if (cu > ma)
            ma = cu, mapo = i;
        if (sTree[i].n < 2)
            return i;
    }
    return mapo;
}

int tot = 1, CNT = 0;
unsigned int startClock;
vector<int> vs;
vector< pair<pair<int,int>,int> > va;
pair<int,int> search (Status s0) {
    sTree[1] = s0;
    tot = 1;
    while (clock()-startClock<TIMELIMIT) {
        CNT++;
        int cur = 1, T = 0, t = 0, win = 0;
        vs.clear();
        va.clear();
        while (1) {
            if (cur == 0)
                break;
            if (!sTree[cur].exd) {
                sTree[cur].getBoard(a);
                findPossiblePos(3 - sTree[cur].color, possibleVec);
                if (!possibleVec.size()) {
                    win = sTree[cur].color;
                    break;
                }
                sTree[cur].exd = 1;
                sTree[cur].lson = tot + 1;
                for (int i = 0, sz = possibleVec.size(); i < sz; i++) {
                    int nx = possibleVec[i].first, ny = possibleVec[i].second;
                    sTree[++tot] = Status();
                    sTree[cur].son.insert(make_pair((short)nx * 9 + ny, tot));
                    sTree[tot].fa = cur;
                    sTree[tot].mvx = nx;
                    sTree[tot].mvy = ny;
                    a[nx][ny] = 3 - sTree[cur].color;
                    sTree[tot].init(3 - sTree[cur].color, a);
                    q1 = q2 = 0;
                    findPossiblePos(sTree[cur].color, tmpvec);
                    sTree[tot].n = sTree[tot].n1 = q1 + q2 + tmpvec.size() / 4;
                    sTree[tot].w = sTree[tot].w1 = q2;
                    a[nx][ny] = 0;
                }
                sTree[cur].rson = tot;
                break;
            } else {
                vs.push_back(cur);
                cur = getBestSon(cur, 0);
                va.push_back(make_pair(make_pair(sTree[cur].mvx, sTree[cur].mvy), sTree[cur].color));
                T++;
            }
        }
        t = T;
        int curColor = 3 - sTree[cur].color;
        sTree[cur].getBoard(a);
        while (1) {
            pair<int,int> pos = findPossiblePos(curColor);
            if (pos.first == -1) {
                win = 3 - curColor;
                break;
            }
            va.push_back(make_pair(make_pair(pos.first, pos.second), curColor));
            t++;
            a[pos.first][pos.second] = curColor;
            curColor = 3 - curColor;
        }

        for (int i = 0; i < T; i++) {
            int uu = vs[i];
            sTree[uu].n++;
            if (sTree[uu].color == win)
                sTree[uu].w++;
            int ff = sTree[uu].fa;
            if (i == 0)
                continue;
            for (int j = i - 1; j < t; j++) {
                if (sTree[ff].color != va[j].second) {
                    int k = sTree[ff].son[(short)(va[j].first.first * 9 + va[j].first.second)];
                    if (k) {
                        sTree[k].n1++;
                        if (sTree[k].color == win)
                            sTree[k].w1++;
                    }
                }
            }
            if (clock() - startClock > TIMELIMIT)
                break;
        }
    }
    int bss = getBestSon(1, 1);
    int rex = sTree[bss].mvx, rey = sTree[bss].mvy;
    for (int i = 1; i <= tot; i++)
        sTree[i] = Status();
    return make_pair(rex, rey);
}

void init() {
    srand(time(NULL));
    for (int i = 0; i <= 200000; i++)
        _beta[i] = sqrt(K / (K + 3 * i));
}

int bd[9][9];
void GetUpdate (std::pair<int, int> location) {
    bd[location.first][location.second] = 2 - ai_side;
}

std::pair<int, int> Action() {
    startClock = clock();
    Status ss = Status();
    ss.init(2 - ai_side, bd);
    ss.getBoard(a);
    findPossiblePos(ai_side + 1);
    for (int i = 0, sz = possibleVec.size(); i < sz; i++) {
        if (possibleVec[i].first == 0 && possibleVec[i].second == 0)
            return bd[0][0] = ai_side + 1, make_pair(0, 0);
        if (possibleVec[i].first == 8 && possibleVec[i].second == 0)
            return bd[8][0] = ai_side + 1, make_pair(8, 0);
        if (possibleVec[i].first == 0 && possibleVec[i].second == 8)
            return bd[0][8] = ai_side + 1, make_pair(0, 8);
        if (possibleVec[i].first == 8 && possibleVec[i].second == 8)
            return bd[8][8] = ai_side + 1, make_pair(8, 8);
    }
    CNT = 0;
    pair<int,int> ret = search(ss);
    bd[ret.first][ret.second] = ai_side + 1;
    return ret;
}
\end{lstlisting}

\subsection{随机模拟}


\subsection{更新节点}

\end{document}